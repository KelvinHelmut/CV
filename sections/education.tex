\cvsect{Educación}

\begin{entrylist}
	\entry
		{2014 -- 2019}
		{Ingeniería de software}
		{Universidad La Salle}
		{
		    Especialidad en investigación, desarrollo e innovación(I+D+I), gracias al conocimiento obtenido sobre estructura de datos, diseño y arquitectura de software, inteligencia artificial(machine learning, deep learning), minería de datos, administración de proyectos, computación distribuida y paralela, entre otros.\\
		    
		    Egresado: Julio - 2019 \\
		    Bachillerato: En proceso
		}
	\entry
		{2009 -- 2011}
		{Computación e informática}
		{Instituto superior UNITEK - IDAT}
		{
		    Especialidad en programación, para ello se requirió el aprendizaje de lógica de programación, algunos lenguajes de programación(C++, Java, .Net) y tecnologías informáticas.\\
		    
		    Egresado: Diciembre - 2011
        }
\end{entrylist}

\newpage

\cvsect{Cursos y MOOCs}
\begin{entrylist}
    \entry
		{15/06/2020}
		{Data Science Math Skills}
		{Duke University - Coursera}
		{
		    Instructor: Daniel Egger y Paul Bendich \\
		    \vspace{-5mm}
    	    \begin{itemize}
    	        \setlength\itemsep{0pt}
    	        \setlength\parskip{0pt}
    	        \item Welcome to Data Science Math Skills
    	        \item Building Blocks for Problem Solving
    	        \item Functions and Graphs
    	        \item Measuring Rates of Change
    	        \item Introduction to Probability Theory
            \end{itemize}
            
            \textbf{Certificado:} \\
            ID: XP73L8MVSYLE \\
            URL: {\href{https://coursera.org/verify/XP73L8MVSYLE}{coursera.org/verify/XP73L8MVSYLE}}
		}
    \entry
		{03/06/2020}
		{IA para todos}
		{deeplearning.ai - Coursera}
		{
		    Instructor: Andrew Ng \\
		    \vspace{-5mm}
    	    \begin{itemize}
    	        \setlength\itemsep{0pt}
    	        \setlength\parskip{0pt}
    	        \item ¿Qué es la IA?
    	        \item Creación de proyectos de IA
    	        \item Desarrollo de IA en su empresa
    	        \item La IA y la sociedad
            \end{itemize}
            
            \textbf{Certificado:} \\
            ID: W5Y25Q7GVVKB \\
            URL: {\href{https://coursera.org/verify/W5Y25Q7GVVKB}{coursera.org/verify/W5Y25Q7GVVKB}}
		}
    \entry
		{03/06/2020}
		{Introduction to Flutter Development Using Dart}
		{App Brewery}
		{
		    Instructor: Angela Yu \\
		    \vspace{-5mm}
    	    \begin{itemize}
    	        \setlength\itemsep{0pt}
    	        \setlength\parskip{0pt}
    	        \item Introduction to Flutter Development
    	        \item Installation and Setup for Flutter Development
    	        \item How to Create Flutter Apps From Scratch (I Am Rich Project)
    	        \item Running Your App on a Physical Device
    	        \item I Am Poor - App Challenge
    	        \item MiCard - How to Build Beautiful UIs with Flutter Widgets
    	        \item Dicee - Building Apps with State
    	        \item Boss Level Challenge 1 - Magic 8 Ball
    	        \item Xylophone - Using Flutter and Dart Packages to Speed Up Development
    	        \item Quizzler - Modularising \& Organising Flutter Code
    	        \item Boss Level Challenge 2 - Destini
            \end{itemize}
            
            \textbf{Certificado:} En anexos
		}
	\entry
		{05/05/2020}
		{Machine Learning}
		{Stanford University - Coursera}
		{
		    Instructor: Andrew Ng \\
		    \vspace{-5mm}
    	    \begin{itemize}
    	        \setlength\itemsep{0pt}
    	        \setlength\parskip{0pt}
    	        \item Introduction
    	        \item Linear Regression with One Variable
    	        \item Linear Algebra Review
    	        \item Linear Regression with Multiple Variables
    	        \item Octave/Matlab Tutorial
    	        \item Logistic Regression
    	        \item Regularization
    	        \item Neural Networks: Representation
    	        \item Neural Networks: Learning
    	        \item Advice for Applying Machine Learning
    	        \item Machine Learning System Design
    	        \item Support Vector Machines
    	        \item Unsupervised Learning
    	        \item Dimensionality Reduction
    	        \item Anomaly Detection
    	        \item Recommender Systems
    	        \item Large Scale Machine Learning
    	        \item Application Example: Photo OCR
    	    \end{itemize}
    	    \textbf{Calificación:} \\
    	    95.30\%
		}
	\entry
		{06/04/2020}
		{Cloud Computing}
		{Escuela de organización industrial - Google Activate}
		{
		    Instructor: Juanjo García y Moisés Navarro \\
		    \vspace{-5mm}
		    \begin{itemize}
		        \setlength\itemsep{0pt}
    	        \setlength\parskip{0pt}
		        \item Introducción: qué es la nube, ventajas e inconvenientes
		        \item Cómo se organiza la nube: desde la infraestructura hasta el software
		        \item Seguridad en Cloud
		        \item Innovación y transformación tecnológica desde el usuario
		        \item Qué necesitan y qué buscan las empresas
		        \item Casos de uso
		        \item Conclusiones
		    \end{itemize}
		    
		    \textbf{Certificado:} \\
		    ID: U7B G5R DZP \\
		    URL: {\href{https://LEARNDIGITAL.WITHGOOGLE.COM/ACTIVATE/validate-certificate-code}{https://LEARNDIGITAL.WITHGOOGLE.COM/ACTIVATE/validate-certificate-code}}
		}
\end{entrylist}